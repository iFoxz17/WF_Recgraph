\chapter{Abstract}
L'\textbf{allineamento di sequenze genomiche} è una parte fondamentale della biologia molecolare moderna, che permette di individuare regioni di somiglianza all'interno delle sequenze nucleotidiche. I progressi degli ultimi anni nelle tecnologie di sequenziamento hanno permesso di avere a disposizione sequenze di lunghezza sempre maggiore, oltre a una quantità notevole di dati; per riuscire a sfruttare a pieno questi progressi tecnologici è necessario riuscire a sviluppare algoritmi di allineamento più veloci ed efficienti, che riescano a scalare con la crescita delle sequenze da allineare. In questo lavoro viene presentato l'\textbf{algoritmo \textit{wavefront}}, un nuovo approccio di programmazione dinamica per l'allineamento di sequenze basato sull'astrazione del \emph{fronte d'onda}: nella prima parte viene riportata la letteratura sul tema dell'allineamento, partendo da coppie di sequenze e successivamente generalizzando a strutture a grafo; nella seconda parte viene presentato l'algoritmo \textbf{\textit{wavefront}}, partendo sempre dal caso più semplice di due sequenze per poi passare a estensioni sui grafi; infine, nella terza parte viene presentato un prototipo realizzato per l'implementazione dell'algoritmo, procedendo con un confronto con \emph{Recgraph}, un tool sviluppato in precedenza dal laboratorio \href{https://algolab.eu/}{BIAS} che effettua allineamento con metodologie standard.